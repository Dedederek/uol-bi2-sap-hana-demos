%%%%%%%%%%%%%%%%%%%%%%%%%%%%%%%%%%%%%%%%%%%%%%%%%%%%%%%%%%%%%%%%%%%%%%%%%%%%%
%
% 26/05/2010
% Created by by Bill Lampos
%
% 26/07/2013
%	Used by Viktor Dmitriyev
%
% Feel free to use (copy) the structure (latex formatting source code)
% but not the content of this document.
%
%%%%%%%%%%%%%%%%%%%%%%%%%%%%%%%%%%%%%%%%%%%%%%%%%%%%%%%%%%%%%%%%%%%%%%%%%%%%%

\documentclass[compress,red,t]{beamer}
%\documentclass[hyperref={pdfpagelabels=false}]{beamer}

\mode<presentation>

% The Beamer class comes with a number of default slide themes
% which change the colors and layouts of slides. Below this is a list
% of all the themes, uncomment each in turn to see what they look like.

%\usetheme{default}
%\usetheme{AnnArbor}
%\usetheme{Antibes}
%\usetheme{Bergen}
%\usetheme{Berkeley}
%\usetheme{Berlin}
%\usetheme{Boadilla}
%\usetheme{CambridgeUS}
%\usetheme{Copenhagen}
%\usetheme{Darmstadt}
%\usetheme{Dresden}
%\usetheme{Frankfurt}
%\usetheme{Goettingen}
%\usetheme{Hannover}
%\usetheme{Ilmenau}
%\usetheme{JuanLesPins}
%\usetheme{Luebeck}
%\usetheme{Madrid}
%\usetheme{Malmoe}
%\usetheme{Marburg}
%\usetheme{Montpellier}
%\usetheme{PaloAlto}
%\usetheme{Pittsburgh}
%\usetheme{Rochester}
%\usetheme{Singapore}
\usetheme{Szeged}
%\usetheme{Warsaw}


% As well as themes, the Beamer class has a number of color themes
% for any slide theme. Uncomment each of these in turn to see how it
% changes the colors of your current slide theme.

\usecolortheme{default}
%\usecolortheme{albatross}
%\usecolortheme{beaver}
%\usecolortheme{beetle}
%\usecolortheme{crane}
%\usecolortheme{dolphin}
%\usecolortheme{dove}
%\usecolortheme{fly}
%\usecolortheme{lily}
%\usecolortheme{orchid}
%\usecolortheme{rose}
%\usecolortheme{seagull}
%\usecolortheme{seahorse}
%\usecolortheme{whale}
%\usecolortheme{wolverine}


\usefonttheme{structurebold}
% font themes: default, professionalfonts, serif, structurebold, structureitalicserif, structuresmallcapsserif

% pdf is displayed in full screen mode automatically
%\hypersetup{pdfpagemode=FullScreen}

% define your own colours:
\definecolor{Red}{rgb}{1,0,0}
\definecolor{Blue}{rgb}{0,0,1}
\definecolor{Green}{rgb}{0,1,0}
\definecolor{magenta}{rgb}{1,0,.6}
\definecolor{lightblue}{rgb}{0,.5,1}
\definecolor{lightpurple}{rgb}{.6,.4,1}
\definecolor{gold}{rgb}{.6,.5,0}
\definecolor{orange}{rgb}{1,0.4,0}
\definecolor{hotpink}{rgb}{1,0,0.5}
\definecolor{newcolor2}{rgb}{.5,.3,.5}
\definecolor{newcolor}{rgb}{0,.3,1}
\definecolor{newcolor3}{rgb}{1,0,.35}
\definecolor{darkgreen1}{rgb}{0, .35, 0}
\definecolor{darkgreen}{rgb}{0, .6, 0}
\definecolor{darkred}{rgb}{.75,0,0}

\xdefinecolor{olive}{cmyk}{0.64,0,0.95,0.4}
\xdefinecolor{purpleish}{cmyk}{0.75,0.75,0,0}

% \usepackage{beamerinnertheme_______}
% inner themes include circles, default, inmargin, rectangles, rounded

%\usepackage{beamerouterthemesmoothbars}
% outer themes include default, infolines, miniframes, shadow, sidebar, smoothbars, smoothtree, split, tree

\useoutertheme[subsection=false]{smoothbars}

% to have the same footer on all slides
%\setbeamertemplate{footline}[text line]{xxx xxx xxx}
%\setbeamertemplate{footline}[text line]{} % or empty footer

% To replace the footer line in all slides with a simple slide count uncomment this line
%\setbeamertemplate{footline}[page number] 

% Current footer with author name & short name of the presentation & page number
\makeatletter
	\setbeamertemplate{footline} {
	  \leavevmode%
	  \hbox{%
	  \begin{beamercolorbox}[wd=.333333\paperwidth,ht=2.25ex,dp=1ex,center]{author in head/foot}%
	    \usebeamerfont{author in head/foot}\insertshortauthor%~~\beamer@ifempty{\insertshortinstitute}{}{(\insertshortinstitute)}
	  \end{beamercolorbox}%
	  \begin{beamercolorbox}[wd=.333333\paperwidth,ht=2.25ex,dp=1ex,center]{title in head/foot}%
	    \usebeamerfont{title in head/foot}\insertshorttitle
	  \end{beamercolorbox}%
	   \begin{beamercolorbox}[wd=.333333\paperwidth,ht=2.25ex,dp=1ex,right]{date in head/foot}%
	     \usebeamerfont{date in head/foot}\date{}\hspace*{2em}
	     \insertframenumber{} / \inserttotalframenumber\hspace*{2ex} 
	   \end{beamercolorbox}}%
	  \vskip0pt%
	}
\makeatother

% Alligment to the bottom
\newenvironment{bottompar}{\par\vspace*{\fill}}{\clearpage}

% Applying underlying to the whole slides' headers in the presentation
\let \oldframetitle \frametitle
\renewcommand{\frametitle}[1] {\oldframetitle { \underline{#1}}}

% include packages
\usepackage{subfigure}
\usepackage{multicol}
\usepackage{amsmath}
\usepackage{epsfig}
\usepackage{graphicx}
\usepackage[all,knot]{xy}
\xyoption{arc}
\usepackage{url}
\usepackage{multimedia}
\usepackage{hyperref}
\usepackage{setspace}

%%%%%%%%%%%%%%%%%%%%%%%%%%%
%
% Start of the content of the presentation.
%
%%%%%%%%%%%%%%%%%%%%%%%%%%%

\title{SAP HANA Tips \& Tricks}
\subtitle{External connection to SAP HANA DB}
\author{Viktor Dmitriyev}
%\institute{\href{mailto:viktor.dmitriyev@uni-oldenburg.de}{viktor.dmitriyev@uni-oldenburg.de}}
\institute{viktor.dmitriyev@uni-oldenburg.de}
\date{Very Large Business Applications \\ University of Oldenburg \\ \vspace{.25cm} July 26, 2013}

\begin{document}

\frame{
	\titlepage
}

\section[Agenda]{}
	\frame{\frametitle{Agenda}
			\tableofcontents
		\href{https://github.com/vdmitriyev/saphana-demos-bi2course-vlba} {github link to demos}
	}

\section{Java \& JDBC}
	\subsection{Connecting to SAP HANA database from JAVA}
		\frame{\frametitle{Connecting to SAP HANA database from JAVA}
			Connecting from Java to the SAP HANA DB:
			\vspace{0.25cm}
			\begin{itemize}
				\item Prepare working environment:
					\begin{itemize}
						\item Install Java JDK (latest version).
						\item Set 'CLASSPATH' to the driver provided by SAP \\ 
							(\emph{can be extracted from publicly available SAP HANA client})
					\end{itemize}
				\item General algorithm:
					\begin{itemize}
						\item Connecting: Create java connection object with correct credentials
						\item Pre-processing: Create prepared statement for sql
						\item Running: Execute query and fetch result
						\item Releasing: Close prepared statement and connection to database
					\end{itemize}
			\end{itemize}
		}

	\subsection{Demo for JDBC usage}
		\frame{\frametitle{Demo for JDBC usage}
			How to run demo example:
			\vspace{0.25cm}
			\begin{itemize}
				\item Start from reading 'README.md'
					\begin{itemize}
						\item For 'Windows': run 'run\_win.bat'
						\item For 'Linux': run './run\_linux' in terminal
					\end{itemize}
			\end{itemize}
		}
		
\section{Python \& ODBC}
	\subsection{Connecting to SAP HANA database from Python}
		\frame{\frametitle{Connecting to SAP HANA database from Python}
			Connecting from Python to the SAP HANA DB:
			\vspace{0.25cm}
			\begin{itemize}
				\item Multiple options are available:
				\begin{itemize}
					\item Option  \#1: Using external ODBC driver (ex: pyodbc)
					\item Option  \#2: Using build-in SAP HANA client (dbapi)
				\end{itemize}
			\end{itemize}
		}
		
	\subsection{Demo for ODBC usage}
		\frame{\frametitle{Demo for python usage}
			How to run demo example:
			\vspace{0.25cm}
			\begin{itemize}
				\item Start from reading 'README.md'
				\item To run option \#1: 'run\_win\_option\_01.bat'
				\item To run option \#2: 'run\_win\_option\_02.bat'
				\item Compare result of outputs from both options
			\end{itemize}
		}
		
\section{MS Excel \& ODBO}
	\subsection{Connecting to SAP HANA database from MS Excel}
		\frame{\frametitle{Connecting to SAP HANA database from MS Excel}
			Connecting from MS Excel to the SAP HANA DB:
			\vspace{0.25cm}
			\begin{itemize}
				\item Most popular BI tool
				\item Uses MDX 32bit driver (cubes)				
			\end{itemize}
		}
		
	\subsection{Demo for ODBO usage}
		\frame{\frametitle{Demo for usage}
			How to run demo example:
			\vspace{0.25cm}
			\begin{itemize}
				\item Start from reading 'sap-hana-tips-excel.pdf'
				\item Prepare SAP HANA instance (create analytical views)
				\item Open MS Excel
				\item Retrieve data
			\end{itemize}
		}
		
\section*{}
	\frame{
		\begin{center}
			\huge
			\begin{bottompar}
				{Thank you for attention!} \\
				\vspace{0.50cm}				
				{Any questions?}
			\end{bottompar}
		\end{center}
	}

\end{document}


